\section{Metodologia utilizada}
\label{section:UsedMethodology}

Foram executados 6 sistemas implementados em Java já utilizados em outro 
trabalho de \cite{polo2009decreasing}.
Os detalhes dos sistemas  relacionados aos números de linhas de código e
o número de casos de testes foram obtidos usando o plugin Source Code 
Metrics para Netbeans \cite{sourceCodeMetrics2016} e são apresentados
junto ao número de FOMs para cada um na tabela 
(\ref{table:QuantitativeSystemDescription})\footnote{Os casos de teste
e os sistemas utilizados neste trabalho estão acessíveis em 
http://bit.do/SUTs4MutationTesting.}.

\begin{table}[h]
  \label{table:QuantitativeSystemDescription}
  \centering
  \caption{Descrição quantitativa sobre os sistemas}
  \begin{tabular}{|c|c|c|c|} \hline
    Aplicação               & Linhas de código  & Número de casos de teste     & Número de FOMs \\ \hline
    Bisect                  & 35                & 25                           & 198            \\ \hline
    Bub                     & 50                & 256                          & 172            \\ \hline
    Find                    & 62                & 135                          & 299            \\ \hline
    Fourballs               & 42                & 96                           & 292            \\ \hline
    Mid                     & 57                & 125                          & 317            \\ \hline
    Triangulo(TryTip)       & 88                & 216                          & 605            \\ \hline
  \end{tabular}
\end{table}

Os testes usados em todos os sistemas foram implementados no seguinte 
trabalho \cite{polo2009decreasing}.

\subsection{Produção dos FOMs e casos de teste}
A geração foi feita usando a ferramenta MuJava \cite{ma2005mujava}, 
a partir dos mutantes gerados foram criados casos de teste usando JUnit, 
uma biblioteca Java para testes. Os casos de teste gerados foram 
executados sobre os mutantes, resultando em uma matriz cuja combinação 
de linha (mutante) e coluna (caso de teste) contém 0 se o mutante 
não foi morto em 1 caso tenha sido morto.

Para a análise, os FOMs (mutantes de primeira ordem) vivos foram considerados como equivalente para
que o \textit{score}\footnote{O \textit{score} é calculado pela relação: 
(mutantes mortos - (total - equivalentes))/(total - equivalentes)} máximo 
seja inicialmente 1.0, facilitando a comparação entre as estratégias.

\subsection{Produção dos SOMs}
As estratégias utilizadas para produção de SOMs (mutantes de segunda ordem) foram implementados por
Polo\cite{polo2009decreasing} and Mateo\cite{reales2013validating}. Das
estratégias avaliadas foram escolhidas: \textit{FirstToLast}, 
\textit{RandomMix} and \textit{Different-Operators}, definias 
respectivamente por \textit{First2Last}, \textit{Random} e
\textit{DiffOp}. Além dessas 3 estratégias citadas , foi escolhida mais
uma de \cite{reales2013validating}: \textit{Each-Choice}.

Essas abordagens demostram uma redução no número de SOMs gerados
mantendo a eficácia dos testes e foram escolhidas pela facilidade
de produção, implementação e bons resultados obtidos.

\subsection{Mutação seletiva}
O objetivo da mutação seletiva é remover os FOMs gerados pelos \textit{x} 
operadores que mais geraram FOMs, onde \textit{x} é um número inteiro. Uma
abordagem que parece promissora pois como os mutantes são dos operadores 
que mais geraram mutantes, uma quantidade significativa de mutantes
será removida, que por sua vez podem ser mutantes que sejam mortos por muitos
casos de uso, e caso isso se mostre verdade, serão removidos uma parcela
considerável de mutantes, reduzindo o custo computacional de processamento
e mantendo o \textit{score} próximo ao original.

\subsubsection{Implementação}
Para implementar a estratégia de mutação seletiva foi criado um código
em Ruby\footnote{Linguagem de programação dinâmica, mais detalhes 
em: https://www.ruby-lang.org/pt/} que recebe como argumentos:
\begin{enumerate}
\item Diretório raíz dos problemas, por exemplo: 'sources/' (Obrigatório).
\item Conjunto com quantos \textit{x} operadores cujos FOMs serão removidos da 
matriz original, por exemplo: '[2,5,7,6]' (Opcional).
\end{enumerate}

Ao executar o código é criado um arquivo chamado MS\_\textit{X}.csv para cada 
\textit{x} passado no segundo argumento, caso especificado. Caso não seja 
passado o segundo argumento, o programa irá retirar os FOMs gerados 
pelos 2, 5 e 10 operadores que mais geraram FOMs e inserílos em arquivos
 MS\_\textit{X}\_removed.csv para que seja possível também analisar o 
complemento da estratégia seletiva.

Após a aplicação da estratégia, foi executado um programa\footnote{esta 
ferramenta foi desenvolvida utilizando um algoritmo genético de busca
[Féderle, Guizzo, Colanzi, Vergilio e Spinosa, 2013]} que tenta encontrar
o conjunto mínimo de caso de testes que mata a maioria dos mutantes, porém 
como este problema é complexo computacionalmete, este programa retorna um 
conjunto próximo do mínimo, porém não determinístico, e por esse motivo é
executado 10 vezes. Os conjuntos resultantes de cada execução são escritos
em arquivos com prefixo 'testcases\_' para cada um dos resultados de cada
estratégia.

Por fim é executado um outro código em escrito Ruby que recebe como 
argumentos:
\begin{enumerate}
\item Diretório raíz dos problemas, por exemplo: 'sources/' (Obrigatório).
\item '-\--summary' Flag para agrupar por média simples aritmética os scores
por estratégia, e não por conjuntos de caso de teste (Opcional).
\end{enumerate}
Que busca na raíz recebida como parâmetro por arquivos com o prefixo 
'\textit{testcases\_}'. Para cada um dos arquivos, lê os conjuntos de 
casos de teste únicos e para cada um calcula os scores na matriz de FOMs
mortos originais obtendo como saída uma média do score para cada um 
dos arquivos.

Os resultados são impressos na saída padrão em JSON\footnote{JSON é uma
formatação leve de troca de dados, mais detalhes em http://www.json.org/}
de cada uma das estratégias comparando-os aos FOMs mortos originais, 
seus resultados serão apresentados a seguir.