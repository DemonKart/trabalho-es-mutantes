\section{Conclusão}
\label{section:conclusao}
Este estudo apresentou a implementação da estratégia de Mutação Seletiva, 
assim como os resultados obtidos e uma análise breve sobre seus resultados
comparando-os aos mutantes de segunda ordem propostos.

Com os resultados apresentados, foi possível ver que algumas das 
estratégias de mutação seletiva apresentadas foram melhores ou no mínimo 
empataram ficaram equiparadas às estratégias de mutantes de segunda ordem.
Entretanto isso pode variar conforme o problema e quantidade de mutantes
gerados. No geral, vimos que a estratégia que obteve melhor desempenho
geral para os problemas propostos foi a MS\_5\_removed, que contém os 
mutantes retirados da mutação seletiva comum.

\section{Considerações Finais e Trabalhos futuros}
\label{section:future-work}
Algo que foi considerado ao fim desse estudo e análise, é que quando 
removemos uma quantidade muito grande de mutantes, o \textit{score}
final acaba sendo prejudicado, como ocorreu com a estratégia MS\_10. 

Algo que pode-se estudar para mitigar esse problema seria ao invés de
remover os mutantes gerados pelos \textit{X} operadores que geram mais
mutantes, calcular quantos operadores foram utilizados para gerar os 
mutantes de primeira ordem e retirar 10\%, 20\% ou 50\% desses operadores,
por exemplo, pois assim teremos a garantia que sempre restarão mutantes 
para serem avaliados.